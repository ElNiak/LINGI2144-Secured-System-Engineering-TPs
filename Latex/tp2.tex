\documentclass{article}
\usepackage[top=3.1cm, bottom=3.1cm, left=2.5cm, right=2.5cm]{geometry}
\usepackage[T1]{fontenc}
\usepackage[utf8]{inputenc}
\usepackage[english]{babel}
\usepackage{graphicx}
\usepackage[toc,page]{appendix} 
\usepackage{eurosym}
\usepackage{gensymb}
\usepackage[dvipsnames]{xcolor}
\usepackage[normal]{caption}
\usepackage{mathtools, bm}
\usepackage{amssymb, bm}
%\usepackage{wrapfig}
\usepackage{floatflt}
\usepackage{enumitem}
\usepackage{MnSymbol,wasysym}
\usepackage[export]{adjustbox}
\usepackage{float}
\usepackage{fancyhdr}
\pagestyle{fancy}
\usepackage{titlesec}
\usepackage{soul}
\usepackage{amsmath,amsfonts,amssymb}
\usepackage{hyperref}
\usepackage{qtree}
%\usepackage{chemfig}
\usepackage{tikz}
\usepackage{pgfplots}
\usepackage{multicol}
\usepackage{multirow}
\usepackage{pgffor}
\usepackage{qtree}
%\usepackage{mhchem}
%\usepackage[demo]{graphicx}
\usepackage{subcaption}
\usepackage{listings}
\usepackage[squaren, Gray, cdot]{SIunits}
\usepackage{inconsolata}
\usepackage{minted}
%\usepackage{syntax} %Fait planter latex pour une raison quelconque

\usepackage{color}
\definecolor{pblue}{rgb}{0.13,0.13,1}
\definecolor{pgreen}{rgb}{0,0.5,0}
\definecolor{pred}{rgb}{0.9,0,0}
\definecolor{pgrey}{rgb}{0.46,0.45,0.48}
\definecolor{mediumslateblue}{rgb}{0.48, 0.41, 0.93}
\definecolor{electricviolet}{rgb}{0.56, 0.0, 1.0}

\newcommand{\bmat}[4]{\begin{bmatrix} #1 & #2 \\ #3 & #4\end{bmatrix}}
\newcommand{\bmatn}[9]{\begin{bmatrix} #1 & #2 & #3\\ #4 & #5 & #6 \\ #7 & #8 & #9\end{bmatrix}}

\renewcommand{\labelitemii}{$\bullet$}
\renewcommand{\labelitemiii}{$\circ$}
%\renewcommand{\labelitemiv}{$\bullet$}


\newcommand{\codecourse}{LINGI2144}
\newcommand{\titlecourse}{Secured System Engineering}
\newcommand{\othor}{\\
\textsc{Crochet} Christophe\\
\textsc{Duchene} Fabien\\
\textsc{Given-Wilson} Thomas\\
\textsc{Strebelle} Sebastien}
\newcommand{\professor}{\textsc{Legay} Axel}
\newcommand{\ayear}{2020 - 2021}
\newcommand{\year}{2020}

\newenvironment{Figure} %for multicols
  {\par\medskip\noindent\minipage{\linewidth}}
  {\endminipage\par\medskip}

\usepackage{listings}

\lstset{
  basicstyle=\ttfamily,
  keywordstyle=\color{pblue},
  keywordstyle=[2]{\color{mediumslateblue}},
  keywordstyle=[3]{\color{electricviolet}},
  identifierstyle=\color{black},
  commentstyle=\itshape\color{pgreen},
  stringstyle=\color{pred},
  language=Java,
  showspaces=false,
  showtabs=false,
  breaklines=true,
  showstringspaces=false,
  breakatwhitespace=true,
  aboveskip=0.3cm,belowskip=0.3cm,
  mathescape=true,
  moredelim=[il][\textcolor{pgrey}]{\$\$},
  moredelim=[is][\textcolor{pgrey}]{\%\%}{\%\%},
  morekeywords={then,end,type,String},
  morekeywords=[2]{invariant,variant,var},
  extendedchars=true,
  literate=
	{á}{{\'a}}1 {é}{{\'e}}1 {í}{{\'i}}1 {ó}{{\'o}}1 {ú}{{\'u}}1
	{Á}{{\'A}}1 {É}{{\'E}}1 {Í}{{\'I}}1 {Ó}{{\'O}}1 {Ú}{{\'U}}1
	{à}{{\`a}}1 {è}{{\`e}}1 {ì}{{\`i}}1 {ò}{{\`o}}1 {ù}{{\`u}}1
	{À}{{\`A}}1 {È}{{\'E}}1 {Ì}{{\`I}}1 {Ò}{{\`O}}1 {Ù}{{\`U}}1
	{ä}{{\"a}}1 {ë}{{\"e}}1 {ï}{{\"i}}1 {ö}{{\"o}}1 {ü}{{\"u}}1
	{Ä}{{\"A}}1 {Ë}{{\"E}}1 {Ï}{{\"I}}1 {Ö}{{\"O}}1 {Ü}{{\"U}}1
	{â}{{\^a}}1 {ê}{{\^e}}1 {î}{{\^i}}1 {ô}{{\^o}}1 {û}{{\^u}}1
	{Â}{{\^A}}1 {Ê}{{\^E}}1 {Î}{{\^I}}1 {Ô}{{\^O}}1 {Û}{{\^U}}1
	{œ}{{\oe}}1 {Œ}{{\OE}}1 {æ}{{\ae}}1 {Æ}{{\AE}}1 {ß}{{\ss}}1
	{ű}{{\H{u}}}1 {Ű}{{\H{U}}}1 {ő}{{\H{o}}}1 {Ő}{{\H{O}}}1
	{ç}{{\c c}}1 {Ç}{{\c C}}1 {ø}{{\o}}1 {å}{{\r a}}1 {Å}{{\r A}}1
	{€}{{\EUR}}1 {£}{{\pounds}}1
}
\pagenumbering{roman}
\title{\codecourse : \titlecourse}
\author{\othor}
\date{September \year}
\fancyhead[R]{\codecourse}

\renewcommand{\footrulewidth}{pt}
\fancyfoot[L]{\codecourse}
\fancyfoot[C]{Page \thepage}
\fancyfoot[R]{\year}

\newcommand{\colR}[1]{\color{red}{#1}}
\newcommand{\colRB}[1]{\color{red}{[#1]}}
\newcommand{\sep}{\ \wedge\ }

\DeclareMathOperator{\fib}{fib}
\DeclareMathOperator{\ok}{ok}
\DeclareMathOperator{\abs}{abs}

\pgfplotsset{compat=1.14}

\begin{document}
        \hfill\includegraphics[scale=0.5]{image/logoepl.png}
        
        \vspace*{\fill}
            
        \begin{center}
        
            \rule{1\textwidth}{1pt}\\
	            \vspace{0.5\baselineskip}
		            \begin{LARGE}
	                	\textbf{\codecourse : \titlecourse}\\
	                	Tutorial 2: Memory Safety
		            \end{LARGE}
		        \vspace{0.5\baselineskip}       
	        \rule{1\textwidth}{1pt}\\
	        
	        \vspace{0.5\baselineskip}
	        
	        \includegraphics[scale=1.5]{image/MCP.jpg}\\

	        \vspace{0.5\baselineskip}
	            Academic year : \ayear\\
                
		\end{center}
		
            \vspace*{\fill}
            
        \begin{tabular}{l@{\hspace{0.0cm}}r}
        
                \begin{minipage}{7cm}\noindent\textbf{Teacher :} \professor\\
                \noindent\textbf{Course :} \codecourse\\
                \noindent\textbf{Collaborators :} \othor 
                \end{minipage}
                &
                
        \end{tabular} 

\newpage

%\tableofcontents

\newpage
\pagenumbering{arabic}
%\begin{itemize} //Bullet points
%    \item [$\bullet$]
%    \item [$\bullet$]
%\end{itemize}

%\begin{multicols}{2} //Multicolonne
%
%\vfill\null
%\columnbreak
%
%\end{multicols}

%\begin{figure}[h]
%    \centering
%    \includegraphics[scale = 0.7]{image/10.PNG}
%    \caption{Titre}
%    \label{fig:titre}
%\end{figure}


\section{Prerequisite}
\noindent Working directory: \lstinline{~/SecurityClass/Tutorial-02}\\


\noindent Connection:
\begin{table}[h!]
\centering
\label{tab:my-table}
\begin{tabular}{c|c}
\textbf{username} & \textbf{password} \\ \hline
admin          & nimda         
\end{tabular}
\end{table}


\section{Exercise}
\subsection{Spacial Bounds}
Let's look at the \lstinline{get-n.c} file. This is a simple utility to return the nth character of a string given to the utility. Let's build and test it under normal usage:

\begin{center}
    \lstinline{gcc -o get-n get-n.c}\\
    \lstinline{./get-n 0123456789 4}\\
    \lstinline{./get-n abcdefghij 8}
\end{center}
\noindent\textbf{What do you think will happen if we give some different numbers as the second argument?}
\begin{itemize}
    \item Hint: Try a few different numbers of very different values, e.g. 100, 1000, 10000, -4.
\end{itemize}
\noindent\textbf{How would you fix this code?}\\

\noindent Now let's look at \lstinline{my-cut.c} that acts like the cut utility by return the characters in a range. We can build and test with
\begin{center}
    \lstinline{gcc -o my-cut my-cut.c}\\
    \lstinline{./my-cut 0123456789 2 7}\\
    \lstinline{./my-cut abcdefghij 3 6}
\end{center}
\noindent\textbf{What do you expect to see now if we give some very different numbers for the second and third arguments?}
\begin{itemize}
    \item Hint: See how much you can see before you cause a segmentation fault.
\end{itemize}
\noindent\textbf{How would you fix this code?}\\

\noindent From this we can see that memory safety is not only a serious concern, but we can even see things well outside the memory space we should be looking at. The examples above only read this space, but we could of course write into this space in most cases, leading to potentially serious vulnerabilities.

\subsection{Use After Free (adapted from lectures)}
Let's revisit the use after free example from the lectures. Let's look at \lstinline{simple.c}. What do we expect to happen if we compile and run it?
\begin{center}
    \lstinline{gcc -o simple simple.c}\\
    \lstinline{./simple}
\end{center}
\noindent When we use the memory after the free we have a segmentation fault (we
cannot use this memory segment, it's been freed)!
But if we add in a new allocation as shown in \lstinline{after-free.c} that can reuse
the same memory segment this changes
\begin{center}
    \lstinline{gcc -o after-free after-free.c}\\
    \lstinline{./after-free}
\end{center}
\noindent Now we are able to inject the bad function behaviour into the "legitimate" function memory space of "\lstinline{malloc1}". As described in the lecture,
the (re)allocation of memory has allowed us to replace the "\lstinline{malloc1}" function with the "bad" function via \lstinline{malloc2}.\\

\noindent This works because the memory is immediately reused. But we can break this by allocating another segment of memory in between. Have a look at the code for \lstinline{again.c}. \textbf{How do you think this will behave? Why?} We can check the behaviour as usual with:
\begin{center}
    \lstinline{gcc -o again again.c}\\
    \lstinline{./again}
\end{center}
\noindent A similar program is \lstinline{yet-again.c} that has a slight change in the allocations of memory. \textbf{What do you expect to happen now? Why?} This can be checked with:
\begin{center}
    \lstinline{gcc -o yet-again yet-again.c}\\
    \lstinline{./yet-again}
\end{center}



\subsection{Double Free}
Let's look at some other ways that freeing memory can be exploited. Observe
that before the same memory was reallocated. (The memory management
doesn't want to keep creating new memory segments, instead giving one
that was "freed" back to the same program again when the program asks for
another small segment of about the same size.)\\


\noindent Look at the code double-free.c that allocates and frees some memory. Observe that the memory allocated to "a" is freed twice. \textbf{What do you think will happen?} We can observe some output with
\begin{center}
    \lstinline{./double-free}
\end{center}
\noindent See that despite never assigning anything to "f" initially, we are still able to
see a well formatted output. Similarly, editing "f" has effects on "d", despite
them (supposedly) being different memory chunks.\\


\noindent WARNING: The binary here is provided since the version of the memory management libraries on the system fixes this problem. (The binary is
statically linked which makes it much bigger than a normal binary.)\\


\noindent We can see that the problem is now fixed on this system by compiling and checking:
\begin{center}
    \lstinline{gcc -o fixed-double-free double-free.c}\\
    \lstinline{./fixed-double-free}
\end{center}

\noindent This is also true if we use another compiler:
\begin{center}
    \lstinline{clang -o clang-double-free double-free.c}\\
    \lstinline{./clang-double-free}
\end{center}
\noindent Here this fix is handled by the memory allocation and management libraries on the system.


\subsection{Uninitialised Pointers and compiler support}
Notice that for 2.2 and 2.3 you were not asked to try and find a solution to this problem. This is because the "solutions" are very compelled or rely on tools and heavy instrumentation (that we will look at later).\\

\noindent Let's look at \lstinline{uninit.c} that defines a pointer and then prints the string at
that pointer.
\begin{center}
    \lstinline{gcc -o uninit uninit.c}\\
    \lstinline{./uninit}
\end{center}
\noindent This prints out some part of memory that we never initialised. Clearly this
is not the "correct" way to use a pointer, but the compiler doesn't care.\\

\noindent This code is very simple, so we can ask the compiler to give us all warnings
and detect this:
\begin{center}
    \lstinline{gcc -Wall -o uninit uninit.c}
\end{center}
\noindent Now we see that the uninitialisd variable "a" is detected by the compiler.
However, we also have a warning about not returning an int from main
which we don't care about.\\

\noindent Let's see if \lstinline{gcc} can find any problems with \lstinline{double-free.c}
\begin{center}
    \lstinline{gcc -Wall -o double-free double-free.c}
\end{center}
\noindent  Here \lstinline{gcc} can detect the uninitialised variables, but not the double free of
"a".\\

\noindent However, this is a limitation of \lstinline{gcc}, other compilers (or tools) can detect this at compile time. Let's look at \lstinline{clang}:
\begin{center}
    \lstinline{clang -Wextra -Wall --analyze -o double-free.xml double-free.c}
\end{center}
\noindent The above starts to show the limitations of current (widely used) tools
for detecting some of the vulnerabilities that we know and understand well.
Of course there are other more advanced (and possibly more expensive) tools
that do better, but most of them will not catch complex vulnerabilities that
combine control 
ow and unexpected conditions with basic vulnerabilities
like we looked at here.

\subsection{\lstinline{NULL} Pointers}
Another approach used in many languages is to not allow any pointer to
be "uninitialized" to random values or addresses. Instead, the language (or
compiler) sets all pointers to the \lstinline{NULL} value (typically 0) when they are
defined.\\

\noindent Let's look at the code in stupid-null.c that re-uses out functions from 2.2
and some function pointers. Clearly the second "bad" function pointer will
be \lstinline{NULL} when we try to call the function. Let's see what this does
\begin{center}
    \lstinline{gcc -o stupid-null stupid-null.c}\\
    \lstinline{./stupid-null}
\end{center}
\noindent As we can see trying to use \lstinline{NULL} as a function doesn't work very well.\\

\noindent This example is stupid and trivial, but notice that even with all warnings \lstinline{gcc} doesn't detect this!
\begin{center}
    \lstinline{gcc -Wall -o stupid-null stupid-null.c}
\end{center}
\noindent We can see that \lstinline{clang} does a little better than \lstinline{gcc} at detecting this:
\begin{center}
    \lstinline{clang -Wall -Wextra --analyze -o stupid-null.xml stupid-null.c}
\end{center}


\noindent Now let's look at some code that is a little more complex in \lstinline{null-use.c}.
\textbf{Can you work out what this code does and how it breaks?} Again we can build and run as usual:
\begin{center}
    \lstinline{gcc -Wall -o null-use null-use.c}\\
    \lstinline{./null-use}
\end{center}
Observe that gcc does not report any warnings for this code, but when we
run it we can see evidence of a memory capture or use after free. Further,
even when an improved "free exploitable" function is used to (better) clean
up on free, this does not prevent a different exploit (\textbf{null} dereference.\\

\noindent Note that in the code for  \lstinline{null-use.c} pointers are generally initialised to
\lstinline{NULL}, and set to \lstinline{NULL} upon free inside objects. Also the printing and
freeing functions check their argument is not \lstinline{NULL} before proceeding. This
is the beginnings of standard coding practices in many languages to try and
avoid the problems illustrated here.\\

\noindent You can check this code with clang and see that the only warning clang can find is in the "free exploitable"
function. Note that clang detects that the argument to "free exploitable"
could be uninitialized, but not that the function pointer in "bad1" could be
uninitialized.
\begin{center}
    \lstinline{clang -Wall -Wextra --analyze -o null-use.xml null-use.c}
\end{center}


\subsection{Open Exercises}
Observe that for 2.2 to 2.5 there is no "solve this" type exercises to do.
The goal of these exercises is for you to understand the simple concepts
underneath many kinds of security vulnerabilities and how they work. In
practice such errors in software are excellent ways to gain access to areas of
memory that should not be available, or to execute code that should not be
executed.\\

\noindent The rest of this tutorial (after this section) has three parts. 2.7 Used a
fuzzing tool and should be able to be completed in the tutorial. 2.8 Lists
some other tools that operate in this space, either offensively or defensively.
Note that if you have \textbf{not} installed them prior to the tutorial, some may
take some time to download and install. Also many of the tools here are not
very stable or reliable, they will only work on specific systems, architectures,
environments, languages, etc. 2.9 Is to explore other languages and try to
cause faults/crashes that could be exploitable. This is open to your choice
of language and expertise.\\

\noindent Note that the rest of this section suggests some potential programs to
create yourself. However, we advise that you start on 2.7 and come back
here when it is running.\\

\noindent Most of the examples shown in 2.1 to 2.5 rely on obvious and straight
forward mistakes. \textbf{Can you combine of obfuscate them so they are hard to
reach or only operate under special conditions?} Some examples to consider:
\begin{enumerate}
    \item Adapt the programs with \lstinline{gcc} or \lstinline{clang} warnings (2.4 & 2.5) to not show any warnings for either compiler.
    \item Cause the bad behaviour from one of the exercises today via an overflow or other unusual condition.
    \item Adapt a program from Tutorial 1 or 2 to only display bad behaviour when a specific input is detected.
    \item After 2.7 today is done, detect a crash from Tutorial 1 with \lstinline{afl}
\end{enumerate}
\newpage
\subsection{Fuzzing with American Fuzzy Lop (\lstinline{afl})}
Let's look at some code that reads a string from input and then acts according
to this string. We can compile and test with
\begin{center}
    \lstinline{gcc -o fuzz-me fuzz-me.c}\\
    \lstinline{echo hello | ./fuzz-me}\\
    \lstinline{echo deadbeef | ./fuzz-me}
\end{center}
\noindent Observe that when we encounter "deadbeef" our code aborts.\\

\noindent Now let's rebuild our code using the American fuzzy lop (\lstinline{afl}) fuzzer. This
approach is when we have access to the source code:\footnote{\url{https://github.com/google/AFL/blob/master/afl-gcc.c}}
\begin{center}
    \lstinline{afl-gcc -o afl-fuzz-me fuzz-me.c}
\end{center}
\noindent This builds an instrumented version of the code for use with a \lstinline{afl}.\\

\noindent Now we can ask \lstinline{afl} to execute the instrumented binary with the input we
provide and produce some output. Let's try a simple example:
\begin{center}
    \lstinline{afl-showmap -o /dev/null -- ./afl-fuzz-me < <(echo hello)}
\end{center}
\noindent We should see some basic output from the program and the information on
tuples captured (but we directed this to \lstinline{/dev/null} so we won't see it).\\

\noindent Aside: tuples are the blocks of code that a
 went through to produce the
output. We won't be looking at them today, but they can be of interest later.\\

\noindent Let's repeat this for some other inputs:
\begin{center}
    \lstinline{afl-showmap -o /dev/null -- ./afl-fuzz-me < <(echo dead)}\\
    \lstinline{afl-showmap -o /dev/null -- ./afl-fuzz-me < <(echo deadbeef)}
\end{center}
Observe that the program output is as we expected and the number of tuples
changes (since we go deeper into the nested conditionals).\\

\noindent Now that we have seen the basics of \lstinline{afl}
 interacting with the instrumented
program, let's try fuzzing our program. Let's create a test case to start our
fuzzing:
\begin{center}
    \lstinline{mkdir fuzz-tests}\\
    \lstinline{echo hello >./fuzz-tests/foo}
\end{center}
\noindent and we will create a directory for our results
\begin{center}
    \lstinline{mkdir fuzz-results}
\end{center}
\noindent We may need to set the following:
\begin{center}
    \lstinline{sudo echo core >/proc/sys/kernel/core_pattern}
\end{center}
\noindent WARNING: the next command may be very slow and very CPU intensive,
if you don't have good battery life or 5-15 minutes to wait, don't do this now!
Now we can start the fuzzer with
\begin{center}
    \lstinline{afl-fuzz -i fuzz-tests -o fuzz-results -- ./afl-fuzz-me}
\end{center}
\noindent This command starts the fuzzer and specifies the inputs, outputs, and the
program to fuzz.
Now we wait (and this may take some time - this is a great moment to
take a break, get coffee, talk to a classmate, etc.)...\\

\noindent Eventually you should see a crash in the a
 output screen. Once this
appears, you can kill a
 with "\lstinline{Ctrl+C}".
Now if we look in

\begin{center}
    \lstinline{cd /fuzz-results/crashes}
\end{center}
\noindent we should find a file containing a string that crashes our program, in this
case "deadbeef"
% \begin{small}
% \medskip
% \bibliographystyle{IEEEtran}
% \bibliography{bib}
% \nocite{*}
% \renewcommand\mkbibnamefamily[1]{\textbf{#1}}
% \end{small}
\end{document}
