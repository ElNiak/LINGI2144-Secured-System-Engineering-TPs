\documentclass{article}
\usepackage[top=3.1cm, bottom=3.1cm, left=2.5cm, right=2.5cm]{geometry}
\usepackage[T1]{fontenc}
\usepackage[utf8]{inputenc}
\usepackage[english]{babel}
\usepackage{graphicx}
\usepackage[toc,page]{appendix} 
\usepackage{eurosym}
\usepackage{gensymb}
\usepackage[dvipsnames]{xcolor}
\usepackage[normal]{caption}
\usepackage{mathtools, bm}
\usepackage{amssymb, bm}
%\usepackage{wrapfig}
\usepackage{floatflt}
\usepackage{enumitem}
\usepackage{MnSymbol,wasysym}
\usepackage[export]{adjustbox}
\usepackage{float}
\usepackage{fancyhdr}
\pagestyle{fancy}
\usepackage{titlesec}
\usepackage{soul}
\usepackage{amsmath,amsfonts,amssymb}
\usepackage{hyperref}
\usepackage{qtree}
%\usepackage{chemfig}
\usepackage{tikz}
\usepackage{pgfplots}
\usepackage{multicol}
\usepackage{multirow}
\usepackage{pgffor}
\usepackage{qtree}
%\usepackage{mhchem}
%\usepackage[demo]{graphicx}
\usepackage{subcaption}
\usepackage{listings}
\usepackage[squaren, Gray, cdot]{SIunits}
\usepackage{inconsolata}
\usepackage{minted}
%\usepackage{syntax} %Fait planter latex pour une raison quelconque

\usepackage{color}
\definecolor{pblue}{rgb}{0.13,0.13,1}
\definecolor{pgreen}{rgb}{0,0.5,0}
\definecolor{pred}{rgb}{0.9,0,0}
\definecolor{pgrey}{rgb}{0.46,0.45,0.48}
\definecolor{mediumslateblue}{rgb}{0.48, 0.41, 0.93}
\definecolor{electricviolet}{rgb}{0.56, 0.0, 1.0}

\newcommand{\bmat}[4]{\begin{bmatrix} #1 & #2 \\ #3 & #4\end{bmatrix}}
\newcommand{\bmatn}[9]{\begin{bmatrix} #1 & #2 & #3\\ #4 & #5 & #6 \\ #7 & #8 & #9\end{bmatrix}}

\renewcommand{\labelitemii}{$\bullet$}
\renewcommand{\labelitemiii}{$\circ$}
%\renewcommand{\labelitemiv}{$\bullet$}


\newcommand{\codecourse}{LINGI2144}
\newcommand{\titlecourse}{Secured System Engineering}
\newcommand{\othor}{\\
\textsc{Crochet} Christophe\\
\textsc{Duchene} Fabien\\
\textsc{Given-Wilson} Thomas\\
\textsc{Strebelle} Sebastien}
\newcommand{\professor}{\textsc{Legay} Axel}
\newcommand{\ayear}{2020 - 2021}
\newcommand{\year}{2020}

\newenvironment{Figure} %for multicols
  {\par\medskip\noindent\minipage{\linewidth}}
  {\endminipage\par\medskip}

\usepackage{listings}

\lstset{
  basicstyle=\ttfamily,
  keywordstyle=\color{pblue},
  keywordstyle=[2]{\color{mediumslateblue}},
  keywordstyle=[3]{\color{electricviolet}},
  identifierstyle=\color{black},
  commentstyle=\itshape\color{pgreen},
  stringstyle=\color{pred},
  language=Java,
  showspaces=false,
  showtabs=false,
  breaklines=true,
  showstringspaces=false,
  breakatwhitespace=true,
  aboveskip=0.3cm,belowskip=0.3cm,
  mathescape=true,
  moredelim=[il][\textcolor{pgrey}]{\$\$},
  moredelim=[is][\textcolor{pgrey}]{\%\%}{\%\%},
  morekeywords={then,end,type,String},
  morekeywords=[2]{invariant,variant,var},
  extendedchars=true,
  literate=
	{á}{{\'a}}1 {é}{{\'e}}1 {í}{{\'i}}1 {ó}{{\'o}}1 {ú}{{\'u}}1
	{Á}{{\'A}}1 {É}{{\'E}}1 {Í}{{\'I}}1 {Ó}{{\'O}}1 {Ú}{{\'U}}1
	{à}{{\`a}}1 {è}{{\`e}}1 {ì}{{\`i}}1 {ò}{{\`o}}1 {ù}{{\`u}}1
	{À}{{\`A}}1 {È}{{\'E}}1 {Ì}{{\`I}}1 {Ò}{{\`O}}1 {Ù}{{\`U}}1
	{ä}{{\"a}}1 {ë}{{\"e}}1 {ï}{{\"i}}1 {ö}{{\"o}}1 {ü}{{\"u}}1
	{Ä}{{\"A}}1 {Ë}{{\"E}}1 {Ï}{{\"I}}1 {Ö}{{\"O}}1 {Ü}{{\"U}}1
	{â}{{\^a}}1 {ê}{{\^e}}1 {î}{{\^i}}1 {ô}{{\^o}}1 {û}{{\^u}}1
	{Â}{{\^A}}1 {Ê}{{\^E}}1 {Î}{{\^I}}1 {Ô}{{\^O}}1 {Û}{{\^U}}1
	{œ}{{\oe}}1 {Œ}{{\OE}}1 {æ}{{\ae}}1 {Æ}{{\AE}}1 {ß}{{\ss}}1
	{ű}{{\H{u}}}1 {Ű}{{\H{U}}}1 {ő}{{\H{o}}}1 {Ő}{{\H{O}}}1
	{ç}{{\c c}}1 {Ç}{{\c C}}1 {ø}{{\o}}1 {å}{{\r a}}1 {Å}{{\r A}}1
	{€}{{\EUR}}1 {£}{{\pounds}}1
}
\pagenumbering{roman}
\title{\codecourse : \titlecourse}
\author{\othor}
\date{September \year}
\fancyhead[R]{\codecourse}

\renewcommand{\footrulewidth}{pt}
\fancyfoot[L]{\codecourse}
\fancyfoot[C]{Page \thepage}
\fancyfoot[R]{\year}

\newcommand{\colR}[1]{\color{red}{#1}}
\newcommand{\colRB}[1]{\color{red}{[#1]}}
\newcommand{\sep}{\ \wedge\ }

\DeclareMathOperator{\fib}{fib}
\DeclareMathOperator{\ok}{ok}
\DeclareMathOperator{\abs}{abs}

\pgfplotsset{compat=1.14}

\begin{document}
        \hfill\includegraphics[scale=0.5]{image/logoepl.png}
        
        \vspace*{\fill}
            
        \begin{center}
        
            \rule{1\textwidth}{1pt}\\
	            \vspace{0.5\baselineskip}
		            \begin{LARGE}
	                	\textbf{\codecourse : \titlecourse}\\
	                	Tutorial 9: Dynamic Analysis
		            \end{LARGE}
		        \vspace{0.5\baselineskip}       
	        \rule{1\textwidth}{1pt}\\
	        
	        \vspace{0.5\baselineskip}
	        
	        \includegraphics[scale=1.5]{image/MCP.jpg}\\

	        \vspace{0.5\baselineskip}
	            Academic year : \ayear\\
                
		\end{center}
		
            \vspace*{\fill}
            
        \begin{tabular}{l@{\hspace{0.0cm}}r}
        
                \begin{minipage}{7cm}\noindent\textbf{Teacher :} \professor\\
                \noindent\textbf{Course :} \codecourse\\
                \noindent\textbf{Collaborators :} \othor 
                \end{minipage}
                &
                
        \end{tabular} 

\newpage

%\tableofcontents

\newpage
\pagenumbering{arabic}
%\begin{itemize} //Bullet points
%    \item [$\bullet$]
%    \item [$\bullet$]
%\end{itemize}

%\begin{multicols}{2} //Multicolonne
%
%\vfill\null
%\columnbreak
%
%\end{multicols}

%\begin{figure}[h]
%    \centering
%    \includegraphics[scale = 0.7]{image/10.PNG}
%    \caption{Titre}
%    \label{fig:titre}
%\end{figure}


\section{Prerequisite}
\noindent Working directory: \lstinline{~/SecurityClass/Tutorial-09}\\


\noindent There is a new image for this tutorial, it is highly recommended you
download this new image to save a lot of effort!
Also note that due to lots of files and sharing between parts of the tutorial,
the tutorial files have not been split up according to section.\\

\noindent If you downloaded the new image and made on changes to your VM configuration then you can skip this section.
If you made changes to your VM configuration it is HIGHLY recommended you keep the memory size small (or change it back to a small
amount).\\

\noindent To run the tutorial today you will need to download some files
these are:
\begin{enumerate}
    \item \lstinline{LiMe} 
    \begin{center}
    \lstinline{git clone https://github.com/504ensicsLabs/LiME.git}
\end{center}
    \item \lstinline{volatility}
    \begin{center}
    \lstinline{git clone https://github.com/volatilityfoundation/volatility.git}
\end{center}
    \item A wannacry memory dump (available from):
    \begin{center}
    \url{https://mega.nz/#!Au5xlCAS!KX5ZJKYzQgDHSa72lPFwqKL6CsZS7oQGbyyQrMTH9XY}
\end{center}
the password to extract this is in the Tutorial files in password.txt.
    \item You will also need some libraries to make things work, you should run:
    \begin{lstlisting}
apt-get install memdump dwarfdump libdistorm3-3 libjansson4 libyara3
    python-distorm3 python-jdcal python-openpyxl python-py python-pytest
    python-yara volatility volatility-tools
    \end{lstlisting}
\end{enumerate}
\noindent After downloaded and installing the above, you should reboot the VM.

\section{Exercise}
\subsection{Memory Dumping with  \lstinline{LiMe}}
Let's build a \lstinline{LiMe} memory profile:
\begin{center}
    \lstinline{cd LiME/src}\\
    \lstinline{make}
\end{center}
\noindent this should build your kernel module, e.g. \lstinline{lime-5.3.0-kali2-686-pae.ko} You can dump the system memory using the \lstinline{LiMe} module you built with (\textbf{take long time}):
\begin{lstlisting}
sudo insmod lime-5.3.0-kali2-686-pae.ko
    "path=/home/admin/SecurityClass/Tutorial-09/memory.dump format=lime"
    \end{lstlisting}
\noindent After the dump has completed you will need to remove the \lstinline{LiMe} module
with
\begin{center}
    \lstinline{sudo rmmod lime}
\end{center}
\noindent The above command (or variations) will be assumed to be used for dumping memory for the rest of this tutorial (but not repeated).
\subsection{Configuring volatility}
NOTE: Details from:
\begin{center}
    \url{https://github.com/volatilityfoundation/volatility/wiki/Installation#getting-volatility}
\end{center}
\noindent but the main steps for Kali Linux 32-bit are described below, you should not
need the above link unless your system is different or something goes wrong.\\

\noindent The parts below are for your own information and configuration, you
should not need to do these on the downloaded image. If the last line of this
section works for you, then skip the configuration here.
The you will need to build a profile. Details are here
\begin{center}
    \url{https://github.com/volatilityfoundation/volatility/wiki/Linux}
\end{center}
\noindent but the quick version that worked for me was and should work on your image is:
\begin{lstlisting}
cd volatility/tools/linux
make
cd ../..
sudo zip ./volatility/plugins/overlays/linux/Kali-5.3.0-686.zip
    ./tools/linux/module.dwarf /boot/System.map-5.3.0-kali2-686-pae
\end{lstlisting}
\noindent you can now find the name of the profile you created with
\begin{center}
    \lstinline{python vol.py --info | grep -i kal}
\end{center}
\noindent Now we can use this \lstinline{LiMe} memory dump and volatility to see which
processes were running when we took the memory dump:
\begin{center}
    \begin{lstlisting}
python vol.py -f /home/admin/SecurityClass/Tutorial-09/memory.dump
    --profile=LinuxKali-5_3_0-686x86 linux_pslist
\end{lstlisting}
\end{center}
\subsection{Memory Dumping Basics}
NOTE: The "\lstinline{build.sh}" script will rebuild the "malware" for this tutorial,
you should not need to do this unless you made major changes to your VM
environment.\\

\noindent In the \lstinline{Tutorial-09/bin} directory you will fond various programs. Run the
program \lstinline{malware_1} with
\begin{center}
    \lstinline{./bin/malware_1}
\end{center}
\noindent which will print \lstinline{"I am evil!!!"} and then pause. During this pause dump the
memory of the system using \lstinline{LiMe}.\\

\noindent Once you have done the memory dump you can kill \lstinline{malware_1} (it is set
to sleep for a long time so you have time to do the dump).\\

\noindent Now we can see which processes were running when we dumped the memory with
\begin{lstlisting}
python vol.py -f /home/admin/SecurityClass/Tutorial-09/memory.dump
    --profile=LinuxKali-5_3_0-686x86 linux_pslist
\end{lstlisting}
\noindent and we should see that malware 1 was running and what process ID it had.\\

\noindent Now we can look inside this process to see that the string "I am evil" is present with
\begin{lstlisting}
python vol.py -f /home/admin/SecurityClass/Tutorial-09/memory.dump
    --profile=LinuxKali-5_3_0-686x86 -p 2452 linux_yarascan -Y "I am evil"
\end{lstlisting}
\noindent where 2452 is the process ID of \lstinline{malware_1} in the memory dump.
\subsection{Memory Dumping Again}
There are four more malware programs in the bin directory. Observe that
only malware 1 is detected as malware by a simple YARA rule:
\begin{center}
    \lstinline{yara evil.yar bin}
\end{center}
\noindent However, the others also print \lstinline{"I am evil!!!"}.\\

\noindent NOTE: You can test your YARA rules from last week on these malware
if you want, some may be detected (or detectable) with YARA.\\

\noindent Run the other programs and perform memory dumps to confirm that you
can observe the malicious behaviour string in the memory.
\subsection{WannaCry}
WARNING: This is malware we are working with, do not use this in any
malicious way! Be very careful how you handle these files!\\

\noindent Extract the dump into a directory, the rest of this section will assume they are in \lstinline{Tutorial-09/wcry}.\\

\noindent First let's use volatility to determine information about the image:
\begin{center}
    \lstinline{python volatility/vol.py -f wcry/wcry.raw imageinfo}
\end{center}
\noindent Now like before we can look at the processes running in the image:
\begin{center}
    \lstinline{python volatility/vol.py -f wcry/wcry.raw pslist}
\end{center}
\noindent Notice in this list a process "\lstinline{@WanaDecryptor@}" that probably has something to do with wannacry.\\

\noindent Let's do a scan of the processes and their parents
\begin{center}
    \lstinline{python volatility/vol.py -f wcry/wcry.raw psscan}
\end{center}
\noindent Here we can see that the wannacry components are being run by the task scheduler "\lstinline{tasksche.exe}".\\

\noindent We can all see what dynamically-linked libraries both the decryptor and the task scheduler are using:
\begin{center}
    \lstinline{python volatility/vol.py -f wcry/wcry.raw dlllist -p 740}\\
    \lstinline{python volatility/vol.py -f wcry/wcry.raw dlllist -p 1940}
\end{center}
\noindent Observe that both are running from the same directory, and one that does
not look like a normal OS directory.\\

\noindent We can also inspect the "handles" (used in Windows APIs for interaction) used by the code:
\begin{center}
    \lstinline{python volatility/vol.py -f wcry/wcry.raw handles -p 1940}
\end{center}
\noindent Here we can see some Mutexes which are interesting as these are often used
to ensure that infection does not occur multiple times.\\

\noindent If we search online for these we will quickly find results that indicate that 
\begin{center}
    \lstinline{MsWinZonesCacheCounterMutexA}
\end{center}
\noindent is used by wannacry.\\

\noindent Now let's extract the network connections that were open during the time
wannacry was running:
\begin{center}
    \lstinline{bulk_extractor -E net -o pcap wcry/wcry.raw}
\end{center}
\noindent This will create several files in the "\lstinline{pcap}" directory for us to use.\\

\noindent Now we can find the IP addresses that the system connected to with
\begin{center}
    \lstinline{tshark -T fields -e ip.src -r pcap/packets.pcap | sort -u}
\end{center}
\noindent This can be useful in later analysis and other parts of program hacking...\\

\noindent Now let's find the files in memory that we care about (running from the
malware's directory):
\begin{lstlisting}
python volatility/vol.py -f wcry/wcry.raw filescan
    | grep ivecuqmanpnirkt615
\end{lstlisting}
\noindent Let's make a diectory for our file and memory dumps:
\begin{center}
    \lstinline{mkdir filedump}\\
    \lstinline{mkdir memdump}
\end{center}
\noindent Now we can see the les we care about we can dry and dump them with
commands like:
\begin{lstlisting}
python volatility/vol.py -f wcry/wcry.raw dumpfiles -Q 0x00000000022ec718
    -D filedump/
\end{lstlisting}
\noindent We can extract the files we're interested in here.
We can also dump the memory of the processes we care about:
\begin{center}
    \lstinline{python volatility/vol.py -f wcry/wcry.raw memdump -p1940,740 -D memdump}
\end{center}
\noindent Now we have the files and the memory when the malware was running.
We can use "strings" to look for interesting strings, and of course explore
further to find more about how the program works and to reverse engineer
the code.\\

\noindent BONUS: Look for interesting strings in the files, can you find the bitcoin
wallet, the IP addresses from the connections, the script sent to the system,
etc?
% \begin{small}
% \medskip
% \bibliographystyle{IEEEtran}
% \bibliography{bib}
% \nocite{*}
% \renewcommand\mkbibnamefamily[1]{\textbf{#1}}
% \end{small}
\end{document}
