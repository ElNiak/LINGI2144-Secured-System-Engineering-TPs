\documentclass{article}
\usepackage[top=3.1cm, bottom=3.1cm, left=2.5cm, right=2.5cm]{geometry}
\usepackage[T1]{fontenc}
\usepackage[utf8]{inputenc}
\usepackage[english]{babel}
\usepackage{graphicx}
\usepackage[toc,page]{appendix} 
\usepackage{eurosym}
\usepackage{gensymb}
\usepackage[dvipsnames]{xcolor}
\usepackage[normal]{caption}
\usepackage{mathtools, bm}
\usepackage{amssymb, bm}
%\usepackage{wrapfig}
\usepackage{floatflt}
\usepackage{enumitem}
\usepackage{MnSymbol,wasysym}
\usepackage[export]{adjustbox}
\usepackage{float}
\usepackage{fancyhdr}
\pagestyle{fancy}
\usepackage{titlesec}
\usepackage{soul}
\usepackage{amsmath,amsfonts,amssymb}
\usepackage{hyperref}
\usepackage{qtree}
%\usepackage{chemfig}
\usepackage{tikz}
\usepackage{pgfplots}
\usepackage{multicol}
\usepackage{multirow}
\usepackage{pgffor}
\usepackage{qtree}
%\usepackage{mhchem}
%\usepackage[demo]{graphicx}
\usepackage{subcaption}
\usepackage{listings}
\usepackage[squaren, Gray, cdot]{SIunits}
\usepackage{inconsolata}
\usepackage{minted}
%\usepackage{syntax} %Fait planter latex pour une raison quelconque

\usepackage{color}
\definecolor{pblue}{rgb}{0.13,0.13,1}
\definecolor{pgreen}{rgb}{0,0.5,0}
\definecolor{pred}{rgb}{0.9,0,0}
\definecolor{pgrey}{rgb}{0.46,0.45,0.48}
\definecolor{mediumslateblue}{rgb}{0.48, 0.41, 0.93}
\definecolor{electricviolet}{rgb}{0.56, 0.0, 1.0}

\newcommand{\bmat}[4]{\begin{bmatrix} #1 & #2 \\ #3 & #4\end{bmatrix}}
\newcommand{\bmatn}[9]{\begin{bmatrix} #1 & #2 & #3\\ #4 & #5 & #6 \\ #7 & #8 & #9\end{bmatrix}}

\renewcommand{\labelitemii}{$\bullet$}
\renewcommand{\labelitemiii}{$\circ$}
%\renewcommand{\labelitemiv}{$\bullet$}


\newcommand{\codecourse}{LINGI2144}
\newcommand{\titlecourse}{Secured System Engineering}
\newcommand{\othor}{\\
\textsc{Crochet} Christophe\\
\textsc{Duchene} Fabien\\
\textsc{Given-Wilson} Thomas\\
\textsc{Strebelle} Sebastien}
\newcommand{\professor}{\textsc{Legay} Axel}
\newcommand{\ayear}{2020 - 2021}
\newcommand{\year}{2020}

\newenvironment{Figure} %for multicols
  {\par\medskip\noindent\minipage{\linewidth}}
  {\endminipage\par\medskip}

\usepackage{listings}

\lstset{
  basicstyle=\ttfamily,
  keywordstyle=\color{pblue},
  keywordstyle=[2]{\color{mediumslateblue}},
  keywordstyle=[3]{\color{electricviolet}},
  identifierstyle=\color{black},
  commentstyle=\itshape\color{pgreen},
  stringstyle=\color{pred},
  language=Java,
  showspaces=false,
  showtabs=false,
  breaklines=true,
  showstringspaces=false,
  breakatwhitespace=true,
  aboveskip=0.3cm,belowskip=0.3cm,
  mathescape=true,
  moredelim=[il][\textcolor{pgrey}]{\$\$},
  moredelim=[is][\textcolor{pgrey}]{\%\%}{\%\%},
  morekeywords={then,end,type,String},
  morekeywords=[2]{invariant,variant,var},
  extendedchars=true,
  literate=
	{á}{{\'a}}1 {é}{{\'e}}1 {í}{{\'i}}1 {ó}{{\'o}}1 {ú}{{\'u}}1
	{Á}{{\'A}}1 {É}{{\'E}}1 {Í}{{\'I}}1 {Ó}{{\'O}}1 {Ú}{{\'U}}1
	{à}{{\`a}}1 {è}{{\`e}}1 {ì}{{\`i}}1 {ò}{{\`o}}1 {ù}{{\`u}}1
	{À}{{\`A}}1 {È}{{\'E}}1 {Ì}{{\`I}}1 {Ò}{{\`O}}1 {Ù}{{\`U}}1
	{ä}{{\"a}}1 {ë}{{\"e}}1 {ï}{{\"i}}1 {ö}{{\"o}}1 {ü}{{\"u}}1
	{Ä}{{\"A}}1 {Ë}{{\"E}}1 {Ï}{{\"I}}1 {Ö}{{\"O}}1 {Ü}{{\"U}}1
	{â}{{\^a}}1 {ê}{{\^e}}1 {î}{{\^i}}1 {ô}{{\^o}}1 {û}{{\^u}}1
	{Â}{{\^A}}1 {Ê}{{\^E}}1 {Î}{{\^I}}1 {Ô}{{\^O}}1 {Û}{{\^U}}1
	{œ}{{\oe}}1 {Œ}{{\OE}}1 {æ}{{\ae}}1 {Æ}{{\AE}}1 {ß}{{\ss}}1
	{ű}{{\H{u}}}1 {Ű}{{\H{U}}}1 {ő}{{\H{o}}}1 {Ő}{{\H{O}}}1
	{ç}{{\c c}}1 {Ç}{{\c C}}1 {ø}{{\o}}1 {å}{{\r a}}1 {Å}{{\r A}}1
	{€}{{\EUR}}1 {£}{{\pounds}}1
}
\pagenumbering{roman}
\title{\codecourse : \titlecourse}
\author{\othor}
\date{September \year}
\fancyhead[R]{\codecourse}

\renewcommand{\footrulewidth}{pt}
\fancyfoot[L]{\codecourse}
\fancyfoot[C]{Page \thepage}
\fancyfoot[R]{\year}

\newcommand{\colR}[1]{\color{red}{#1}}
\newcommand{\colRB}[1]{\color{red}{[#1]}}
\newcommand{\sep}{\ \wedge\ }

\DeclareMathOperator{\fib}{fib}
\DeclareMathOperator{\ok}{ok}
\DeclareMathOperator{\abs}{abs}

\pgfplotsset{compat=1.14}

\begin{document}
        \hfill\includegraphics[scale=0.5]{image/logoepl.png}
        
        \vspace*{\fill}
            
        \begin{center}
        
            \rule{1\textwidth}{1pt}\\
	            \vspace{0.5\baselineskip}
		            \begin{LARGE}
	                	\textbf{\codecourse : \titlecourse}\\
	                	Tutorial 7: Evading Stack Protection \& Format String
		            \end{LARGE}
		        \vspace{0.5\baselineskip}       
	        \rule{1\textwidth}{1pt}\\
	        
	        \vspace{0.5\baselineskip}
	        
	        \includegraphics[scale=1.5]{image/MCP.jpg}\\

	        \vspace{0.5\baselineskip}
	            Academic year : \ayear\\
                
		\end{center}
		
            \vspace*{\fill}
            
        \begin{tabular}{l@{\hspace{0.0cm}}r}
        
                \begin{minipage}{7cm}\noindent\textbf{Teacher :} \professor\\
                \noindent\textbf{Course :} \codecourse\\
                \noindent\textbf{Collaborators :} \othor 
                \end{minipage}
                &
                
        \end{tabular} 

\newpage

%\tableofcontents

\newpage
\pagenumbering{arabic}
%\begin{itemize} //Bullet points
%    \item [$\bullet$]
%    \item [$\bullet$]
%\end{itemize}

%\begin{multicols}{2} //Multicolonne
%
%\vfill\null
%\columnbreak
%
%\end{multicols}

%\begin{figure}[h]
%    \centering
%    \includegraphics[scale = 0.7]{image/10.PNG}
%    \caption{Titre}
%    \label{fig:titre}
%\end{figure}


\section{Prerequisite}
\noindent Working directory: \lstinline{~/SecurityClass/Tutorial-07}\\


\noindent Connection:
\begin{table}[h!]
\centering
\label{tab:my-table}
\begin{tabular}{c|c}
\textbf{username} & \textbf{password} \\ \hline
admin          & nimda         
\end{tabular}
\end{table}
\subsection{Environment Configuration}
On most Linux systems and with most compilers there are protections built
in to prevent various exploits. For today's tutorial we may have to turn some
of these off.\\

\noindent One is the randomisation of memory segments by the Linux kernel. We
can see the current value with
\begin{center}
    \lstinline{sudo cat /proc/sys/kernel/randomize_va_space}
\end{center}
\noindent This is "2" by default on Kali Linux (and most Linux systems). To turn this off for the rest of the sessions by setting the value to "0" we can run
\begin{center}
    \lstinline{echo 0 | sudo tee /proc/sys/kernel/randomize_va_space}\\
    \lstinline{sudo cat /proc/sys/kernel/randomize_va_space}
\end{center}
\noindent The compiler can also insert various protections into code that is compiled.
For the stack, \lstinline{gcc} includes some stack protection by default on many
versions. We can force this to be turned off with the argument
\begin{center}
    \lstinline{-fno-stack-protector}
\end{center}
\noindent At times we may also wish to force gcc to compile code with executable
instructions on the stack, we can enable this with
\begin{center}
    \lstinline{-z execstack}
\end{center}

\section{Exercise}
\subsection{Format String}
Here you will find a program "\lstinline{example.c}" that has a format string vulnerability in the \lstinline{printf} statement that you can use.\\

\noindent This is a program designed to help you explore how to abuse format string
and pass in parameters. You can build the program with:
\begin{center}
    \lstinline{gcc -o example example.c}
\end{center}
\noindent and then run it with format string arguments, for example:
\begin{center}
    \lstinline{./example "%x %x %x %x"}
\end{center}
\noindent Observe that there is a value called "\lstinline{test_value}" that you \textbf{may} be able to modify.\\

\noindent NOTE: Due to memory layout and your architecture it MAY NOT be
easy or possible to inject a format string (e.g. your address may require a
\lstinline{\x00} which would break the formatting). For the rest of this section, you can also use \lstinline{format1.c} or \lstinline{format2.c} if these have better memory layout for you (e.g. you want to change values on the stack).\\


\noindent The goal now is to use a format string to change the value of either
"\lstinline{test_value}" in \lstinline{example.c}, or "\lstinline{target}" in \lstinline{format1.c} or \lstinline{format2.c}.\\

\noindent HINT: The format string component \lstinline{%n} writes back from format string
into an address. You can use this to change the value of one of the "arguments" to the function using your format string.
% \begin{small}
% \medskip
% \bibliographystyle{IEEEtran}
% \bibliography{bib}
% \nocite{*}
% \renewcommand\mkbibnamefamily[1]{\textbf{#1}}
% \end{small}
\end{document}
